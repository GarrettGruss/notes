\documentclass[14pt]{extarticle}
\usepackage[utf8]{inputenc}
\usepackage[T1]{fontenc}
\usepackage{setspace}
\usepackage[margin=1in]{geometry}
\usepackage{hyperref}

\onehalfspacing

\title{Event-Driven Process Mining for Failure Cascade Detection in Vehicle Telemetry Systems}
\author{Garrett Gruss}
\date{\today}

\begin{document}

\section*{Abstract}

This paper provides a brief summary of cloud computing, its history, and various tradeoffs. The distinguishing characteristics of cloud computing, the services and deployment models, and the security challenges are analyzed.

\section{Introducion}

Cloud computing refers to the integration of distributed computing, network storage, load balance, and high availability. These resources can be provisioned on-demand by users to rapidly scale up or down usage.

\section{History}

The concepts of cloud computing originated in the 60s under the term "utility computing", where compute, storage, and applications could be rented for a defined time. The concept of "Grid Computing" also originated in this time to describe a group of computers coupled to form a virtual machine, often to distribute a complex task into multiple subtasks.

Salesforce was the first major provider of software as a service (SaaS), delivering their enterprise software as a service over the internet.

Amazon Web Services (AWS) launched in 2006, providing infrastructure-as-a-service (IaaS) where organizations could provision compute, storage, and network on-demand. By 2007, universities had begun to provide training on cloud computing, offering courses developed in partnership with Google and IBM.

In 2010, the open source community in partnership with Rackspace and NASAS developed a free private cloud system called OpenStack, allowing organizations to standup a private cloud within their network.

\section{Essential Characteristics}

Cloud computing is defined by the national institute of Standards and Technology (NIST) with five core characteristics: On-Demand Self-Service, Broad Network Access, Resource Pooling, Rapid Elasticity, and Measured Service.

\section{Service Models}

Infrastructure as a service (IaaS): Customers get raw compute and storage to build on
Platform as a service (PaaS): Customers get a platform to dev on
Software as a service (SaaS): Customers provided software over web

\section{Deployment Models}

- **Public cloud**: Amazon, Azure, Google 
- **Private Cloud**: Cloud within an org, higher security as network is internal
- **Community Cloud**: Private cloud shared between organizations
- **Hybrid cloud deployment**: Public + Private cloud networked together

\section{Security Considerations}

## Cloud Security
Cloud vendors should encrypt stored passwords and seperate usernames/passwords to guard in the event of a breach.

- **Passwords**: used to connect to cloud services
- **Access Recovery**: Recover access to cloud
- **Encryption**: Use strong password encryption
- **Password Management**: Encourage strong passwords
- **Multi-factor auth**: Biometric, RSA, or USB key
- **Login Monitor**: Monitor for suspicious logins
- **Personal Devices**: Customers should only login on known safe devices to avoid key loggers
- **Virus, Malware, and Trojans**: Hacker attacks user machine to gain access to their cloud account

\section{Impact on Software Development}

\section{Conclusion}



\end{document}